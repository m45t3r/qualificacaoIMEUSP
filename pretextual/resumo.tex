% ---------------------------------------------------------------------------- %
% Resumo
\chapter*{Resumo}

\noindent \citacaoautor. \textbf{\titulotrabalho}. 
\ano. \paginas~f.
\tipotrabalho~(\tipoprograma) - Instituto de Matem�tica e Estat�stica,
Universidade de S�o Paulo, S�o Paulo, \ano.
\\

Elemento obrigat�rio, constitu�do de uma sequ�ncia de frases concisas e
objetivas, em forma de texto.  Deve apresentar os objetivos, m�todos empregados,
resultados e conclus�es.  O resumo deve ser redigido em par�grafo �nico, conter
no m�ximo 500 palavras e ser seguido dos termos representativos do conte�do do
trabalho (palavras-chave). 
Texto texto texto texto texto texto texto texto texto texto texto texto texto
texto texto texto texto texto texto texto texto texto texto texto texto texto
texto texto texto texto texto texto texto texto texto texto texto texto texto
texto texto texto texto texto texto texto texto texto texto texto texto texto
texto texto texto texto texto texto texto texto texto texto texto texto texto
texto texto texto texto texto texto texto texto.
Texto texto texto texto texto texto texto texto texto texto texto texto texto
texto texto texto texto texto texto texto texto texto texto texto texto texto
texto texto texto texto texto texto texto texto texto texto texto texto texto
texto texto texto texto texto texto texto texto texto texto texto texto texto
texto texto.
\\

\noindent \textbf{Palavras-chave:} palavra-chave1, palavra-chave2, palavra-chave3.
