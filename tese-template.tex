% Arquivo LaTeX template de dissertação/tese a ser apresentados à CPG do IME-USP
% 
% Versão 5: Sex Mar  9 18:05:40 BRT 2012
%
% Criação: Jesús P. Mena-Chalco
% Revisão: Fabio Kon e Paulo Feofiloff
% Modificado por Rodrigo Campiolo
%  
% Obs: Leia previamente o texto do arquivo README.md

\documentclass[openany,11pt,twoside,a4paper]{book}

% Arquivo LaTeX de exemplo de dissertação/tese a ser apresentados à CPG do IME-USP
% 
% Versão 5: Sex Mar  9 18:05:40 BRT 2012
%
% Criação: Jesús P. Mena-Chalco
% Revisão: Fabio Kon e Paulo Feofiloff
% Modificado por Rodrigo Campiolo
%  
% Obs: Leia previamente o texto do arquivo README.md

% ---------------------------------------------------------------------------- %
% Pacotes 
\usepackage[T1]{fontenc}
\usepackage[brazil]{babel}
\usepackage[utf8]{inputenc}
\usepackage[pdftex]{graphicx}           % usamos arquivos pdf/png como figuras
\usepackage{setspace}                   % espaçamento flexível
\usepackage{indentfirst}                % indentação do primeiro parágrafo
\usepackage{makeidx}                    % índice remissivo
\usepackage[nottoc]{tocbibind}  	% acrescentamos a bibliografia/indice/conteudo no Table of Contents
\usepackage{courier}                    % usa o Adobe Courier no lugar de Computer Modern Typewriter
\usepackage{type1cm}            	% fontes realmente escaláveis
\usepackage{listings}                   % para formatar código-fonte (ex. em Java)
\usepackage{titletoc}
%\usepackage[bf,small,compact]{titlesec} % cabeçalhos dos títulos: menores e compactos
\usepackage[fixlanguage]{babelbib}
\usepackage[font=small,format=plain,labelfont=bf,up,textfont=it,up]{caption}
\usepackage[usenames,svgnames,dvipsnames]{xcolor}
\usepackage[a4paper,top=2.54cm,bottom=2.0cm,left=2.0cm,right=2.54cm]{geometry} % margens
%\usepackage[pdftex,plainpages=false,pdfpagelabels,pagebackref,colorlinks=true,citecolor=black,linkcolor=black,urlcolor=black,filecolor=black,bookmarksopen=true]{hyperref} % links em preto
\usepackage[pdftex,plainpages=false,pdfpagelabels,pagebackref,colorlinks=true,citecolor=DarkGreen,linkcolor=NavyBlue,urlcolor=DarkRed,filecolor=green,bookmarksopen=true]{hyperref} % links coloridos
\usepackage[all]{hypcap}                    % soluciona o problema com o hyperref e capitulos
\usepackage[round,sort,nonamebreak]{natbib} % citação bibliográfica textual(plainnat-ime.bst)
\bibpunct{(}{)}{;}{a}{\hspace{-0.7ex},}{,} % estilo de citação. Veja alguns exemplos em http://merkel.zoneo.net/Latex/natbib.php

\fontsize{60}{62}\usefont{OT1}{cmr}{m}{n}{\selectfont}

% ---------------------------------------------------------------------------- %
% Cabeçalhos similares ao TAOCP de Donald E. Knuth
\usepackage{fancyhdr}
\pagestyle{fancy}
\fancyhf{}
\renewcommand{\chaptermark}[1]{\markboth{\MakeUppercase{#1}}{}}
\renewcommand{\sectionmark}[1]{\markright{\MakeUppercase{#1}}{}}
\renewcommand{\headrulewidth}{0pt}

% ---------------------------------------------------------------------------- %
\graphicspath{{./figuras/}}             % caminho das figuras (recomendável)
\frenchspacing                          % arruma o espaço: id est (i.e.) e exempli gratia (e.g.) 
\urlstyle{same}                         % URL com o mesmo estilo do texto e não mono-spaced
\makeindex                              % para o índice remissivo
\raggedbottom                           % para não permitir espaços extra no texto
\fontsize{60}{62}\usefont{OT1}{cmr}{m}{n}{\selectfont}
\cleardoublepage
\normalsize

% ---------------------------------------------------------------------------- %
% Opções de listing usados para o código fonte
% Ref: http://en.wikibooks.org/wiki/LaTeX/Packages/Listings
\lstset{ %
language=Java,                  % choose the language of the code
basicstyle=\footnotesize,       % the size of the fonts that are used for the code
numbers=left,                   % where to put the line-numbers
numberstyle=\footnotesize,      % the size of the fonts that are used for the line-numbers
stepnumber=1,                   % the step between two line-numbers. If it's 1 each line will be numbered
numbersep=5pt,                  % how far the line-numbers are from the code
showspaces=false,               % show spaces adding particular underscores
showstringspaces=false,         % underline spaces within strings
showtabs=false,                 % show tabs within strings adding particular underscores
frame=single,	                % adds a frame around the code
framerule=0.6pt,
tabsize=2,	                % sets default tabsize to 2 spaces
captionpos=b,                   % sets the caption-position to bottom
breaklines=true,                % sets automatic line breaking
breakatwhitespace=false,        % sets if automatic breaks should only happen at whitespace
escapeinside={\%*}{*)},         % if you want to add a comment within your code
backgroundcolor=\color[rgb]{1.0,1.0,1.0}, % choose the background color.
rulecolor=\color[rgb]{0.8,0.8,0.8},
extendedchars=true,
xleftmargin=10pt,
xrightmargin=10pt,
framexleftmargin=10pt,
framexrightmargin=10pt
inputencoding=utf8,             % utf8 for brazillian portuguese
extendedchars=true,
literate=%
{é}{{\'{e}}}1
{è}{{\`{e}}}1
{ê}{{\^{e}}}1
{ë}{{\¨{e}}}1
{É}{{\'{E}}}1
{Ê}{{\^{E}}}1
{û}{{\^{u}}}1
{ù}{{\`{u}}}1
{â}{{\^{a}}}1
{à}{{\`{a}}}1
{á}{{\'{a}}}1
{ã}{{\~{a}}}1
{Á}{{\'{A}}}1
{Â}{{\^{A}}}1
{Ã}{{\~{A}}}1
{ç}{{\c{c}}}1
{Ç}{{\c{C}}}1
{õ}{{\~{o}}}1
{ó}{{\'{o}}}1
{ô}{{\^{o}}}1
{Õ}{{\~{O}}}1
{Ó}{{\'{O}}}1
{Ô}{{\^{O}}}1
{î}{{\^{i}}}1
{Î}{{\^{I}}}1
{í}{{\'{i}}}1
{Í}{{\~{Í}}}1
}

% cabeçalho para as páginas das seções anteriores ao capítulo 1 (frontmatter)
\newcommand{\configurapretextual}{
\frontmatter 
\fancyhead[RO]{{\footnotesize\rightmark}\hspace{2em}\thepage}
\setcounter{tocdepth}{2}
\fancyhead[LE]{\thepage\hspace{2em}\footnotesize{\leftmark}}
\fancyhead[RE,LO]{}
\fancyhead[RO]{{\footnotesize\rightmark}\hspace{2em}\thepage}
\onehalfspacing  % espaçamento
}

% Capítulos do trabalho
\newcommand{\configuracapitulos}{
\mainmatter
% cabeçalho para as páginas de todos os capítulos
\fancyhead[RE,LO]{\thesection}
\singlespacing              % espaçamento simples
%\onehalfspacing            % espaçamento um e meio
}

\newcommand{\configurapostextual}{
\renewcommand{\chaptermark}[1]{\markboth{\MakeUppercase{\appendixname\ \thechapter}} {\MakeUppercase{##1}} }
\fancyhead[RE,LO]{}
\appendix
}


%% Configuração de glossário
\usepackage[portuguese]{nomencl}
\usepackage[acronym,nomain,nonumberlist,nopostdot,nohypertypes={acronym}]{glossaries}
\renewcommand{\glossarymark}[1]{}
\makeglossaries
\glossarystyle{super}		% estilo do glossário
\renewcommand*{\glsgroupskip}{} % espaço entre linhas
\renewcommand*{\acronymname}{Lista de Abreviaturas} 	%Altera título da página
%\renewcommand*{\glssettoctitle}{Lista de Abreviaturas}  %Configura o sumário  para o título
%

%%% FIM das configurações

\newcommand{\titulotrabalho} {
    Uma abordagem colaborativa e distribuída para a detecção antecipada de incidentes de segurança
}

\newcommand{\titulotrabalhoingles} {
    Title
}

\newcommand{\autor} {Rodrigo Campiolo}
\newcommand{\citacaoautor} {CAMPIOLO, R.}
\newcommand{\emailautor} {rcampiolo@utfpr.edu.br}

% Definir macros para o nome da Instituição, da Faculdade, etc.
\newcommand{\tipotrabalho}{Tese}   %ou Dissertação ou Qualificação
\newcommand{\universidade}{Universidade de São Paulo}
\newcommand{\programa}{Pós-Graduação em Ciência da Computação}
\newcommand{\faculdade}{Faculdade de Ciência da Computação}
\newcommand{\instituto}{Instituto de Matemática e Estatística}
\newcommand{\titulacao}{Doutor} % ou Mestre
\newcommand{\tipoprograma}{Doutorado} % ou Mestrado
\newcommand{\proforientador}{Prof. Dr. Daniel Macêdo Batista}
\newcommand{\profcoorientador}{Prof. Dr. Coorientador}
\newcommand{\profbancaa}{Prof. Dr. NONAME}
\newcommand{\profbancab}{Prof. Dr. NONAME}
\newcommand{\profbancac}{Prof. Dr. NONAME}
\newcommand{\profbancaafac}{IME-USP} 
\newcommand{\profbancabfac}{IME-USP} 
\newcommand{\profbancacfac}{IME-USP} 
\newcommand{\auxiliofinanceiro}{CAPES/CNPq/FAPESP}

\newcommand{\mes}{fevereiro}
\newcommand{\ano}{2014}
\newcommand{\paginas}{x}

% ---------------------------------------------------------------------------- %
\begin{document}

\configurapretextual

% ---------------------------------------------------------------------------- %
% CAPA
% Nota: O título para as dissertações/teses do IME-USP devem caber em um 
% orifício de 10,7cm de largura x 6,0cm de altura que há na capa fornecida pela SPG.
\thispagestyle{empty}
\begin{center}
    \vspace*{2.3cm}
    \textbf{\Large{\titulotrabalho}}\\
    
    \vspace*{1.2cm}
    \Large{\autor}
    
    \vskip 2cm
    \textsc{
    \tipotrabalho~apresentada\\[-0.25cm] 
    ao\\[-0.25cm]
    \instituto\\[-0.25cm]
    da\\[-0.25cm]
    \universidade\\[-0.25cm]
    para\\[-0.25cm]
    obtenção do título\\[-0.25cm]
    de\\[-0.25cm]
    \titulacao~em Ciências}
    
    \vskip 1.5cm
    Programa: \programa\\
    Orientador: \proforientador\\
    Coorientador: \profcoorientador

    \vskip 1cm
    \normalsize{Durante o desenvolvimento deste trabalho o autor recebeu auxílio
    financeiro da \auxiliofinanceiro}
    
    \vskip 0.5cm
    \normalsize{São Paulo, \mes~de \ano}
\end{center}


% ---------------------------------------------------------------------------- %
% P�gina de rosto (S� PARA A VERS�O DEPOSITADA - ANTES DA DEFESA)
% Resolu��o CoPGr 5890 (20/12/2010)
%
% IMPORTANTE:
%   Coloque um '%' em todas as linhas
%   desta p�gina antes de compilar a vers�o
%   final, corrigida, do trabalho
%
%
\newpage
\thispagestyle{empty}
    \begin{center}
        \vspace*{2.3 cm}
        \textbf{\Large{\titulotrabalho}}\\
        \vspace*{2 cm}
    \end{center}

    \vskip 2cm

    \begin{flushright}
	Esta � a vers�o original da \MakeLowercase{\tipotrabalho} elaborada pelo\\
	candidato (\autor), tal como \\
	submetida � Comiss�o Julgadora.
    \end{flushright}

\pagebreak
 
%% ---------------------------------------------------------------------------- %
% P�gina de rosto (S� PARA A VERS�O CORRIGIDA - AP�S DEFESA)
% Resolu��o CoPGr 5890 (20/12/2010)
%
% Nota: O t�tulo para as disserta��es/teses do IME-USP devem caber em um 
% orif�cio de 10,7cm de largura x 6,0cm de altura que h� na capa fornecida pela SPG.
%
% IMPORTANTE:
%   Coloque um '%' em todas as linhas desta
%   p�gina antes de compilar a vers�o do trabalho que ser� entregue
%   � Comiss�o Julgadora antes da defesa
%
%
\newpage
\thispagestyle{empty}
    \begin{center}
        \vspace*{2.3 cm}
        \textbf{\Large{T�tulo do trabalho a ser apresentado � \\
        CPG para a disserta��o/tese}}\\
        \vspace*{2 cm}
    \end{center}

    \vskip 2cm

    \begin{flushright}
	Esta vers�o da \MakeLowercase{\tipotrabalho} cont�m as corre��es e altera��es sugeridas\\
	pela Comiss�o Julgadora durante a defesa da vers�o original do trabalho,\\
	realizada em 14/12/2010. Uma c�pia da vers�o original est� dispon�vel no\\
	Instituto de Matem�tica e Estat�stica da Universidade de S�o Paulo.
    \end{flushright}
    \vskip 6.2cm

    \begin{quote}
    \noindent Comiss�o Julgadora:
    
    \begin{itemize}
		\item \proforientador~(orientador) - IME-USP %[sem ponto final]
		\item \profbancaa~- \profbancaafac %[sem ponto final]
		\item \profbancab~- \profbancabfac %[sem ponto final]
		\item \profbancac~- \profbancacfac %[sem ponto final]
    \end{itemize}
      
    \end{quote}
\pagebreak
  %somente após a banca e correções

\pagenumbering{roman}     % inicia numeração romana

% ---------------------------------------------------------------------------- %
% Agradecimentos:
% Se o candidato não quer fazer agradecimentos, deve simplesmente eliminar esta página 
\chapter*{Agradecimentos}
Texto texto texto texto texto texto texto texto texto texto texto texto texto
texto texto texto texto texto texto texto texto texto texto texto texto texto
texto texto texto texto texto texto texto texto texto texto texto texto texto
texto texto texto texto. Texto opcional.

% ---------------------------------------------------------------------------- %
% Resumo
\chapter*{Resumo}

\noindent \citacaoautor. \textbf{\titulotrabalho}. 
\ano. \paginas~f.
\tipotrabalho~(\tipoprograma) - Instituto de Matem�tica e Estat�stica,
Universidade de S�o Paulo, S�o Paulo, \ano.
\\

Elemento obrigat�rio, constitu�do de uma sequ�ncia de frases concisas e
objetivas, em forma de texto.  Deve apresentar os objetivos, m�todos empregados,
resultados e conclus�es.  O resumo deve ser redigido em par�grafo �nico, conter
no m�ximo 500 palavras e ser seguido dos termos representativos do conte�do do
trabalho (palavras-chave). 
Texto texto texto texto texto texto texto texto texto texto texto texto texto
texto texto texto texto texto texto texto texto texto texto texto texto texto
texto texto texto texto texto texto texto texto texto texto texto texto texto
texto texto texto texto texto texto texto texto texto texto texto texto texto
texto texto texto texto texto texto texto texto texto texto texto texto texto
texto texto texto texto texto texto texto texto.
Texto texto texto texto texto texto texto texto texto texto texto texto texto
texto texto texto texto texto texto texto texto texto texto texto texto texto
texto texto texto texto texto texto texto texto texto texto texto texto texto
texto texto texto texto texto texto texto texto texto texto texto texto texto
texto texto.
\\

\noindent \textbf{Palavras-chave:} palavra-chave1, palavra-chave2, palavra-chave3.

% ---------------------------------------------------------------------------- %
% Abstract
\chapter*{Abstract}
\noindent \citacaoautor. \textbf{\titulotrabalhoingles}. 
\ano. \paginas~f.
\tipotrabalho (\tipoprograma) - Instituto de Matem�tica e Estat�stica,
Universidade de S�o Paulo, S�o Paulo, \ano.
\\


Elemento obrigat�rio, elaborado com as mesmas caracter�sticas do resumo em
l�ngua portuguesa. De acordo com o Regimento da P�s- Gradua��o da USP (Artigo
99), deve ser redigido em ingl�s para fins de divulga��o. 
Text text text text text text text text text text text text text text text text
text text text text text text text text text text text text text text text text
text text text text text text text text text text text text text text text text
text text text text text text text text text text text text.
Text text text text text text text text text text text text text text text text
text text text text text text text text text text text text text text text text
text text text.
\\

\noindent \textbf{Keywords:} keyword1, keyword2, keyword3.



\tableofcontents          % gera o sumário

% ---------------------------------------------------------------------------- %
%\acrlong{label} - acronimo/sigla longo
%\acrshort{label} - acronimo/sigla curta
%\Gls{TCP} - sigla com o significado primeiro em Maiusculo
%\GLS{TCP} - sigla com o significado tudo em MAIUSCULO
%\gls{TCP} - sigla com o significado tudo em minusculo

%\newglossaryentry{led}{name=LED,description={light-emitting diode},first={light-emitting diode (LED)}}

\newacronym{EWS}{EWS}{\emph{Early Warning System}}
\newacronym{ANATEL}{ANATEL}{Agência Nacional de Telecomunicações}

\printglossaries
\addcontentsline{toc}{chapter}{Lista de Abreviaturas}
 % usando glossaries
\include{pretextual/listasimbolos}

% ---------------------------------------------------------------------------- %
% Listas de figuras e tabelas geradas automaticamente
\listoffigures            
\listoftables            

% ---------------------------------------------------------------------------- %
\configuracapitulos

\include{capitulos/cap-introducao}        % associado ao arquivo: 'cap-introducao.tex'
\include{capitulos/cap-conceitos}         % associado ao arquivo: 'cap-conceitos.tex'
\include{capitulos/cap-conclusoes}        % associado ao arquivo: 'cap-conclusoes.tex'

% ---------------------------------------------------------------------------- %

% cabeçalho para os apêndices
\configurapostextual

\include{postextual/ape-conjuntos}      	% associado ao arquivo: 'ape-conjuntos.tex'

% bibliografia
\backmatter \singlespacing   			% espaçamento simples
\bibliographystyle{bibliografia/plainnat-ime}   % citação bibliográfica textual (pode-se usar o "alpha-ime")
\bibliography{bibliografia/bibliografia}        % associado ao arquivo: 'bibliografia.bib'

% indice remissivo
% �ndice remissivo
\index{TBP|see{periodicidade regi�o codificante}}
\index{DSP|see{processamento digital de sinais}}
\index{STFT|see{transformada de Fourier de tempo reduzido}}
\index{DFT|see{transformada discreta de Fourier}}
\index{Fourier!transformada|see{transformada de Fourier}}

\printindex   % imprime o �ndice remissivo no documento 


\end{document}
