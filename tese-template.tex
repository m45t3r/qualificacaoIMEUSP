% Arquivo LaTeX template de dissertação/tese a ser apresentados à CPG do IME-USP
% 
% Versão 5: Sex Mar  9 18:05:40 BRT 2012
%
% Criação: Jesús P. Mena-Chalco
% Revisão: Fabio Kon e Paulo Feofiloff
% Modificado por Rodrigo Campiolo
%  
% Obs: Leia previamente o texto do arquivo README.txt

\documentclass[openany,11pt,twoside,a4paper]{book}

% Arquivo LaTeX de exemplo de dissertação/tese a ser apresentados à CPG do IME-USP
% 
% Versão 5: Sex Mar  9 18:05:40 BRT 2012
%
% Criação: Jesús P. Mena-Chalco
% Revisão: Fabio Kon e Paulo Feofiloff
% Modificado por Rodrigo Campiolo
%  
% Obs: Leia previamente o texto do arquivo README.md

% ---------------------------------------------------------------------------- %
% Pacotes 
\usepackage[T1]{fontenc}
\usepackage[brazil]{babel}
\usepackage[utf8]{inputenc}
\usepackage[pdftex]{graphicx}           % usamos arquivos pdf/png como figuras
\usepackage{setspace}                   % espaçamento flexível
\usepackage{indentfirst}                % indentação do primeiro parágrafo
\usepackage{makeidx}                    % índice remissivo
\usepackage[nottoc]{tocbibind}  	% acrescentamos a bibliografia/indice/conteudo no Table of Contents
\usepackage{courier}                    % usa o Adobe Courier no lugar de Computer Modern Typewriter
\usepackage{type1cm}            	% fontes realmente escaláveis
\usepackage{listings}                   % para formatar código-fonte (ex. em Java)
\usepackage{titletoc}
%\usepackage[bf,small,compact]{titlesec} % cabeçalhos dos títulos: menores e compactos
\usepackage[fixlanguage]{babelbib}
\usepackage[font=small,format=plain,labelfont=bf,up,textfont=it,up]{caption}
\usepackage[usenames,svgnames,dvipsnames]{xcolor}
\usepackage[a4paper,top=2.54cm,bottom=2.0cm,left=2.0cm,right=2.54cm]{geometry} % margens
%\usepackage[pdftex,plainpages=false,pdfpagelabels,pagebackref,colorlinks=true,citecolor=black,linkcolor=black,urlcolor=black,filecolor=black,bookmarksopen=true]{hyperref} % links em preto
\usepackage[pdftex,plainpages=false,pdfpagelabels,pagebackref,colorlinks=true,citecolor=DarkGreen,linkcolor=NavyBlue,urlcolor=DarkRed,filecolor=green,bookmarksopen=true]{hyperref} % links coloridos
\usepackage[all]{hypcap}                    % soluciona o problema com o hyperref e capitulos
\usepackage[round,sort,nonamebreak]{natbib} % citação bibliográfica textual(plainnat-ime.bst)
\bibpunct{(}{)}{;}{a}{\hspace{-0.7ex},}{,} % estilo de citação. Veja alguns exemplos em http://merkel.zoneo.net/Latex/natbib.php

\fontsize{60}{62}\usefont{OT1}{cmr}{m}{n}{\selectfont}

% ---------------------------------------------------------------------------- %
% Cabeçalhos similares ao TAOCP de Donald E. Knuth
\usepackage{fancyhdr}
\pagestyle{fancy}
\fancyhf{}
\renewcommand{\chaptermark}[1]{\markboth{\MakeUppercase{#1}}{}}
\renewcommand{\sectionmark}[1]{\markright{\MakeUppercase{#1}}{}}
\renewcommand{\headrulewidth}{0pt}

% ---------------------------------------------------------------------------- %
\graphicspath{{./figuras/}}             % caminho das figuras (recomendável)
\frenchspacing                          % arruma o espaço: id est (i.e.) e exempli gratia (e.g.) 
\urlstyle{same}                         % URL com o mesmo estilo do texto e não mono-spaced
\makeindex                              % para o índice remissivo
\raggedbottom                           % para não permitir espaços extra no texto
\fontsize{60}{62}\usefont{OT1}{cmr}{m}{n}{\selectfont}
\cleardoublepage
\normalsize

% ---------------------------------------------------------------------------- %
% Opções de listing usados para o código fonte
% Ref: http://en.wikibooks.org/wiki/LaTeX/Packages/Listings
\lstset{ %
language=Java,                  % choose the language of the code
basicstyle=\footnotesize,       % the size of the fonts that are used for the code
numbers=left,                   % where to put the line-numbers
numberstyle=\footnotesize,      % the size of the fonts that are used for the line-numbers
stepnumber=1,                   % the step between two line-numbers. If it's 1 each line will be numbered
numbersep=5pt,                  % how far the line-numbers are from the code
showspaces=false,               % show spaces adding particular underscores
showstringspaces=false,         % underline spaces within strings
showtabs=false,                 % show tabs within strings adding particular underscores
frame=single,	                % adds a frame around the code
framerule=0.6pt,
tabsize=2,	                % sets default tabsize to 2 spaces
captionpos=b,                   % sets the caption-position to bottom
breaklines=true,                % sets automatic line breaking
breakatwhitespace=false,        % sets if automatic breaks should only happen at whitespace
escapeinside={\%*}{*)},         % if you want to add a comment within your code
backgroundcolor=\color[rgb]{1.0,1.0,1.0}, % choose the background color.
rulecolor=\color[rgb]{0.8,0.8,0.8},
extendedchars=true,
xleftmargin=10pt,
xrightmargin=10pt,
framexleftmargin=10pt,
framexrightmargin=10pt
inputencoding=utf8,             % utf8 for brazillian portuguese
extendedchars=true,
literate=%
{é}{{\'{e}}}1
{è}{{\`{e}}}1
{ê}{{\^{e}}}1
{É}{{\'{E}}}1
{Ê}{{\^{E}}}1
{û}{{\^{u}}}1
{ú}{{\'{u}}}1
{ù}{{\`{u}}}1
{Û}{{\^{U}}}1
{Ú}{{\'{U}}}1
{â}{{\^{a}}}1
{à}{{\`{a}}}1
{á}{{\'{a}}}1
{ã}{{\~{a}}}1
{Á}{{\'{A}}}1
{Â}{{\^{A}}}1
{Ã}{{\~{A}}}1
{ç}{{\c{c}}}1
{Ç}{{\c{C}}}1
{õ}{{\~{o}}}1
{ó}{{\'{o}}}1
{ô}{{\^{o}}}1
{Õ}{{\~{O}}}1
{Ó}{{\'{O}}}1
{Ô}{{\^{O}}}1
{î}{{\^{i}}}1
{í}{{\'{i}}}1
{Î}{{\^{I}}}1
{Í}{{\~{Í}}}1
}

% cabeçalho para as páginas das seções anteriores ao capítulo 1 (frontmatter)
\newcommand{\configurapretextual}{
\frontmatter 
\fancyhead[RO]{{\footnotesize\rightmark}\hspace{2em}\thepage}
\setcounter{tocdepth}{2}
\fancyhead[LE]{\thepage\hspace{2em}\footnotesize{\leftmark}}
\fancyhead[RE,LO]{}
\fancyhead[RO]{{\footnotesize\rightmark}\hspace{2em}\thepage}
\onehalfspacing  % espaçamento
}

% Capítulos do trabalho
\newcommand{\configuracapitulos}{
\mainmatter
% cabeçalho para as páginas de todos os capítulos
\fancyhead[RE,LO]{\thesection}
\singlespacing              % espaçamento simples
%\onehalfspacing            % espaçamento um e meio
}

\newcommand{\configurapostextual}{
\renewcommand{\chaptermark}[1]{\markboth{\MakeUppercase{\appendixname\ \thechapter}} {\MakeUppercase{##1}} }
\fancyhead[RE,LO]{}
\appendix
}


%% Configuração de glossário
\usepackage[portuguese]{nomencl}
\usepackage[acronym,nomain,nonumberlist,nopostdot,nohypertypes={acronym}]{glossaries}
\renewcommand{\glossarymark}[1]{}
\makeglossaries
\glossarystyle{super}		% estilo do glossário
\renewcommand*{\glsgroupskip}{} % espaço entre linhas
\renewcommand*{\acronymname}{Lista de Abreviaturas} 	%Altera título da página
%\renewcommand*{\glssettoctitle}{Lista de Abreviaturas}  %Configura o sumário  para o título
%

%%% FIM das configurações

\newcommand{\titulotrabalho} {
    Uma abordagem colaborativa e distribuída para a detecção antecipada de incidentes de segurança
}

\newcommand{\titulotrabalhoingles} {
    Title
}

\newcommand{\autor} {Rodrigo Campiolo}
\newcommand{\citacaoautor} {CAMPIOLO, R.}
\newcommand{\emailautor} {rcampiolo@utfpr.edu.br}

% Definir macros para o nome da Instituição, da Faculdade, etc.
\newcommand{\tipotrabalho}{Tese}   %ou Dissertação ou Qualificação
\newcommand{\universidade}{Universidade de São Paulo}
\newcommand{\programa}{Pós-Graduação em Ciência da Computação}
\newcommand{\faculdade}{Faculdade de Ciência da Computação}
\newcommand{\instituto}{Instituto de Matemática e Estatística}
\newcommand{\titulacao}{Doutor} % ou Mestre
\newcommand{\tipoprograma}{Doutorado} % ou Mestrado
\newcommand{\proforientador}{Prof. Dr. Daniel Macêdo Batista}
\newcommand{\profcoorientador}{Prof. Dr. Coorientador}
\newcommand{\profbancaa}{Prof. Dr. NONAME}
\newcommand{\profbancab}{Prof. Dr. NONAME}
\newcommand{\profbancac}{Prof. Dr. NONAME}
\newcommand{\profbancaafac}{IME-USP} 
\newcommand{\profbancabfac}{IME-USP} 
\newcommand{\profbancacfac}{IME-USP} 
\newcommand{\auxiliofinanceiro}{CAPES/CNPq/FAPESP}

\newcommand{\mes}{fevereiro}
\newcommand{\ano}{2014}
\newcommand{\paginas}{x}

% ---------------------------------------------------------------------------- %
\begin{document}

\configurapretextual

% ---------------------------------------------------------------------------- %
% CAPA
% Nota: O título para as dissertações/teses do IME-USP devem caber em um 
% orifício de 10,7cm de largura x 6,0cm de altura que há na capa fornecida pela SPG.
\thispagestyle{empty}
\begin{center}
    \vspace*{2.3cm}
    \textbf{\Large{\titulotrabalho}}\\
    
    \vspace*{1.2cm}
    \Large{\autor}
    
    \vskip 2cm
    \textsc{
    \tipotrabalho~apresentada\\[-0.25cm] 
    ao\\[-0.25cm]
    \instituto\\[-0.25cm]
    da\\[-0.25cm]
    \universidade\\[-0.25cm]
    para\\[-0.25cm]
    obtenção do título\\[-0.25cm]
    de\\[-0.25cm]
    \titulacao~em Ciências}
    
    \vskip 1.5cm
    Programa: \programa\\
    Orientador: \proforientador\\
    Coorientador: \profcoorientador

    \vskip 1cm
    \normalsize{Durante o desenvolvimento deste trabalho o autor recebeu auxílio
    financeiro da \auxiliofinanceiro}
    
    \vskip 0.5cm
    \normalsize{São Paulo, \mes~de \ano}
\end{center}


% ---------------------------------------------------------------------------- %
% Página de rosto (SÓ PARA A VERSÃO DEPOSITADA - ANTES DA DEFESA)
% Resolução CoPGr 5890 (20/12/2010)
%
% IMPORTANTE:
%   Coloque um '%' em todas as linhas
%   desta página antes de compilar a versão
%   final, corrigida, do trabalho
%
%
\newpage
\thispagestyle{empty}
    \begin{center}
        \vspace*{2.3 cm}
        \textbf{\Large{\titulotrabalho}}\\
        \vspace*{2 cm}
    \end{center}

    \vskip 2cm

    \begin{flushright}
	Esta é a versão original da \MakeLowercase{\tipotrabalho} elaborada pelo\\
	candidato (\autor), tal como \\
	submetida à Comissão Julgadora.
    \end{flushright}

\pagebreak
 
%% ---------------------------------------------------------------------------- %
% Página de rosto (SÓ PARA A VERSÃO CORRIGIDA - APÓS DEFESA)
% Resolução CoPGr 5890 (20/12/2010)
%
% Nota: O título para as dissertações/teses do IME-USP devem caber em um 
% orifício de 10,7cm de largura x 6,0cm de altura que há na capa fornecida pela SPG.
%
% IMPORTANTE:
%   Coloque um '%' em todas as linhas desta
%   página antes de compilar a versão do trabalho que será entregue
%   à Comissão Julgadora antes da defesa
%
%
\newpage
\thispagestyle{empty}
    \begin{center}
        \vspace*{2.3 cm}
        \textbf{\Large{Título do trabalho a ser apresentado à \\
        CPG para a dissertação/tese}}\\
        \vspace*{2 cm}
    \end{center}

    \vskip 2cm

    \begin{flushright}
	Esta versão da \MakeLowercase{\tipotrabalho} contém as correções e alterações sugeridas\\
	pela Comissão Julgadora durante a defesa da versão original do trabalho,\\
	realizada em 14/12/2010. Uma cópia da versão original está disponível no\\
	Instituto de Matemática e Estatística da Universidade de São Paulo.
    \end{flushright}
    \vskip 6.2cm

    \begin{quote}
    \noindent Comissão Julgadora:
    
    \begin{itemize}
		\item \proforientador~(orientador) - IME-USP %[sem ponto final]
		\item \profbancaa~- \profbancaafac %[sem ponto final]
		\item \profbancab~- \profbancabfac %[sem ponto final]
		\item \profbancac~- \profbancacfac %[sem ponto final]
    \end{itemize}
      
    \end{quote}
\pagebreak
  %somente após a banca e correções

\pagenumbering{roman}     % inicia numeração romana

% ---------------------------------------------------------------------------- %
% Agradecimentos:
% Se o candidato não quer fazer agradecimentos, deve simplesmente eliminar esta página 
\chapter*{Agradecimentos}
Texto texto texto texto texto texto texto texto texto texto texto texto texto
texto texto texto texto texto texto texto texto texto texto texto texto texto
texto texto texto texto texto texto texto texto texto texto texto texto texto
texto texto texto texto. Texto opcional.

% ---------------------------------------------------------------------------- %
% Resumo
\chapter*{Resumo}

\noindent \citacaoautor. \textbf{\titulotrabalho}. 
\ano. \paginas~f.
\tipotrabalho~(\tipoprograma) - Instituto de Matemática e Estatística,
Universidade de São Paulo, São Paulo, \ano.
\\

Elemento obrigatório, constituído de uma sequência de frases concisas e
objetivas, em forma de texto.  Deve apresentar os objetivos, métodos empregados,
resultados e conclusões.  O resumo deve ser redigido em parágrafo único, conter
no máximo 500 palavras e ser seguido dos termos representativos do conteúdo do
trabalho (palavras-chave). 
Texto texto texto texto texto texto texto texto texto texto texto texto texto
texto texto texto texto texto texto texto texto texto texto texto texto texto
texto texto texto texto texto texto texto texto texto texto texto texto texto
texto texto texto texto texto texto texto texto texto texto texto texto texto
texto texto texto texto texto texto texto texto texto texto texto texto texto
texto texto texto texto texto texto texto texto.
Texto texto texto texto texto texto texto texto texto texto texto texto texto
texto texto texto texto texto texto texto texto texto texto texto texto texto
texto texto texto texto texto texto texto texto texto texto texto texto texto
texto texto texto texto texto texto texto texto texto texto texto texto texto
texto texto.
\\

\noindent \textbf{Palavras-chave:} palavra-chave1, palavra-chave2, palavra-chave3.

% ---------------------------------------------------------------------------- %
% Abstract
\chapter*{Abstract}
\noindent \citacaoautor. \textbf{\titulotrabalhoingles}. 
\ano. \paginas~f.
\tipotrabalho (\tipoprograma) - Instituto de Matemática e Estatística,
Universidade de São Paulo, São Paulo, \ano.
\\


Elemento obrigatório, elaborado com as mesmas características do resumo em
língua portuguesa. De acordo com o Regimento da Pós- Graduação da USP (Artigo
99), deve ser redigido em inglês para fins de divulgação. 
Text text text text text text text text text text text text text text text text
text text text text text text text text text text text text text text text text
text text text text text text text text text text text text text text text text
text text text text text text text text text text text text.
Text text text text text text text text text text text text text text text text
text text text text text text text text text text text text text text text text
text text text.
\\

\noindent \textbf{Keywords:} keyword1, keyword2, keyword3.



\tableofcontents          % gera o sumário

% ---------------------------------------------------------------------------- %
%\acrlong{label} - acronimo/sigla longo
%\acrshort{label} - acronimo/sigla curta
%\Gls{TCP} - sigla com o significado primeiro em Maiusculo
%\GLS{TCP} - sigla com o significado tudo em MAIUSCULO
%\gls{TCP} - sigla com o significado tudo em minusculo

%\newglossaryentry{led}{name=LED,description={light-emitting diode},first={light-emitting diode (LED)}}

\newacronym{EWS}{EWS}{\emph{Early Warning System}}
\newacronym{ANATEL}{ANATEL}{Agência Nacional de Telecomunicações}

\printglossaries
\addcontentsline{toc}{chapter}{Lista de Abreviaturas}
 % usando glossaries
\chapter{Lista de Símbolos}
\begin{tabular}{ll}
        $\omega$    & Frequência angular\\
        $\psi$      & Função de análise \emph{wavelet}\\
        $\Psi$      & Transformada de Fourier de $\psi$\\
\end{tabular}


% ---------------------------------------------------------------------------- %
% Listas de figuras e tabelas geradas automaticamente
\listoffigures            
\listoftables            

% ---------------------------------------------------------------------------- %
\configuracapitulos

%% ------------------------------------------------------------------------- %%
\chapter{Introdução}
\label{cap:introducao}

Escrever bem é uma arte que exige muita técnica e dedicação. Há vários bons livros
sobre como escrever uma boa dissertação ou tese. Um dos trabalhos pioneiros e mais
conhecidos nesse sentido é o livro de \citet{eco:09} intitulado 
\emph{Como se faz uma tese}; é uma leitura bem interessante mas, como foi escrito 
em 1977 e é voltado para teses de graduação na Itália, não se aplica tanto a nós.

Para a escrita de textos em Ciência da Computação, o livro de Justin Zobel, 
\emph{Writing for Computer Science} \citep{zobel:04} é uma leitura obrigatória. 
O livro \emph{Metodologia de Pesquisa para Ciência da Computação} de 
\citet{waz:09} também merece uma boa lida.
Já para a área de Matemática, dois livros recomendados são o de Nicholas Higham,
\emph{Handbook of Writing for Mathematical Sciences} \citep{Higham:98} e o do criador
do \TeX, Donald Knuth, juntamente com Tracy Larrabee e Paul Roberts, 
\emph{Mathematical Writing} \citep{Knuth:96}.

O uso desnecessário de termos em lingua estrangeira deve ser evitado. No entanto,
quando isso for necessário, os termos devem aparecer \emph{em itálico}.

\begin{small}
\begin{verbatim}
Modos de citação:
indesejável: [AF83] introduziu o algoritmo ótimo.
indesejável: (Andrew e Foster, 1983) introduziram o algoritmo ótimo.
certo : Andrew e Foster introduziram o algoritmo ótimo [AF83].
certo : Andrew e Foster introduziram o algoritmo ótimo (Andrew e Foster, 1983).
certo : Andrew e Foster (1983) introduziram o algoritmo ótimo.
\end{verbatim}
\end{small}

Uma prática recomendável na escrita de textos é descrever as legendas das
figuras e tabelas em forma auto-contida: as legendas devem ser razoavelmente
completas, de modo que o leitor possa entender a figura sem ler o texto onde a
figura ou tabela é citada.  

Apresentar os resultados de forma simples, clara e completa é uma tarefa que
requer inspiração. Nesse sentido, o livro de \citet{tufte01:visualDisplay},
\emph{The Visual Display of Quantitative Information}, serve de ajuda na
criação de figuras que permitam entender e interpretar dados/resultados de forma
eficiente.

% \emph{Thesis are random access. Do NOT feel obliged to read a thesis from beginning to end.}



%% ------------------------------------------------------------------------- %%
\section{Considerações Preliminares}
\label{sec:consideracoes_preliminares}

Considerações preliminares\footnote{Nota de rodapé (não abuse).}\index{genoma!projetos}.
% index permite acrescentar um item no indice remissivo
Texto texto texto texto texto texto texto texto texto texto texto texto texto
texto texto texto texto texto texto texto texto texto texto texto texto texto
texto texto texto texto texto texto texto.
 

%% ------------------------------------------------------------------------- %%
\section{Objetivos}
\label{sec:objetivo}

Texto texto texto texto texto texto texto texto texto texto texto texto texto
texto texto texto texto texto texto texto texto texto texto texto texto texto
texto texto texto texto texto texto.

%% ------------------------------------------------------------------------- %%
\section{Contribuições}
\label{sec:contribucoes}

As principais contribuições deste trabalho são as seguintes:

\begin{itemize}
  \item Item 1. Texto texto texto texto texto texto texto texto texto texto
  texto texto texto texto texto texto texto texto texto texto.

  \item Item 2. Texto texto texto texto texto texto texto texto texto texto
  texto texto texto texto texto texto texto texto texto texto.

\end{itemize}

%% ------------------------------------------------------------------------- %%
\section{Organização do Trabalho}
\label{sec:organizacao_trabalho}

No Capítulo~\ref{cap:conceitos}, apresentamos os conceitos ... Finalmente, no
Capítulo~\ref{cap:conclusoes} discutimos algumas conclusões obtidas neste
trabalho. Analisamos as vantagens e desvantagens do método proposto ... 

As sequências testadas no trabalho estão disponíveis no Apêndice \ref{ape:sequencias}.
\gls{EWS} e teste \gls{ANATEL}. A \gls{ANATEL}

        % associado ao arquivo: 'cap-introducao.tex'
%% ------------------------------------------------------------------------- %%
\chapter{Conceitos}
\label{cap:conceitos}

Texto texto texto texto texto texto texto texto texto texto texto texto texto
texto texto texto texto texto texto texto texto texto texto texto texto texto
texto texto texto texto texto texto texto texto texto texto texto texto texto
texto texto texto texto texto texto texto texto texto texto texto texto texto
texto texto texto texto texto texto.

%% ------------------------------------------------------------------------- %%
\section{Fundamentos}\index{área do trabalho!fundamentos}
\label{sec:fundamentos}

Texto texto texto texto texto texto texto texto texto texto texto texto texto
texto texto texto texto texto texto texto texto texto texto texto texto texto
texto texto texto texto texto texto texto texto texto texto texto texto texto
texto texto texto texto texto texto texto texto texto texto texto texto texto
texto texto texto.

%% ------------------------------------------------------------------------- %%
\subsection{Ácidos Nucléicos}\index{ácido!nucléico}\index{nucleotídeos}
\label{sec:acidos_nucleicos}

Na Figura~\ref{fig:humanbeta} texto texto texto texto texto texto texto texto
texto texto texto texto texto texto texto texto texto texto texto texto texto
texto texto texto texto texto texto texto texto texto texto texto texto texto
texto texto texto texto texto texto texto texto texto texto texto texto texto
texto texto texto.

\begin{figure}[!h]
  \centering
  \includegraphics[width=.40\textwidth]{graph} 
  \caption{Descrição da figura mostrada.}
  \label{fig:humanbeta} 
\end{figure}

%% ------------------------------------------------------------------------- %%
\subsection{Aminoácidos}\index{ácido!amino|(}
\label{sec:amino_acidos}

Veja na Tabela \ref{tab:amino_acidos}...  texto texto texto texto texto texto
texto texto texto texto texto texto texto texto texto texto texto texto texto
texto texto texto texto texto texto texto texto texto texto texto texto texto
texto texto texto texto texto texto texto texto texto texto texto texto texto
texto texto texto texto texto texto texto texto texto texto texto.

\begin{table}[!t]
\begin{center}
    \begin{tabular}{c|c|l}
	 \hline
	 Código & Abreviatura & Nome completo \\ \hline
     \texttt{A} & Ala & Alanina \\
     \texttt{C} & Cys & Cisteína \\
     ...        & ... & ... \\
     \texttt{W} & Trp & Tiptofano \\
     \texttt{Y} & Tyr & Tirosina \\ \hline
    \end{tabular}
  \caption{Códigos, abreviaturas e nomes dos aminoácidos.}
  \label{tab:amino_acidos}
\end{center}
\end{table}
\index{ácido!amino|)}

Texto texto texto texto texto texto texto texto texto texto texto texto texto
texto texto texto texto texto texto texto texto texto texto texto texto texto
texto texto texto texto texto texto texto texto texto texto texto texto texto
texto texto texto texto texto texto texto texto texto texto texto texto texto
texto texto texto texto texto texto texto.


%% ------------------------------------------------------------------------- %%
\section{Exemplo de Código-Fonte em Java}
\label{sec:exemplo_codigo_fonte}
Texto texto texto texto texto texto texto texto texto texto texto texto texto
texto texto texto texto texto texto texto texto texto texto texto texto texto
texto texto texto texto texto texto texto texto texto texto texto texto texto
texto texto texto texto texto texto texto.

% Foi utilizado o pacote listing para formatar código fonte
% http://ctan.org/tex-archive/macros/latex/contrib/listings/listings.pdf
% Veja no preambulo do arquivo tese-exemplo.tex os parâmetros de configuração.

\begin{lstlisting}[frame=trbl]
    for(i = 0; i < 20; i++)
    {
        // Comentário 
        System.out.println("Mensagem...");
    }
\end{lstlisting}


%% ------------------------------------------------------------------------- %%
\section{Algumas Referências}
\label{sec:algumas_referencias}

É muito recomendável a utilização de arquivos \emph{bibtex} para o gerenciamento
de referências a trabalhos. Nesse sentido existem três plataformas gratuitas
que permitem a busca de referências acadêmicas em formato bib: 
\begin{itemize}
	\item \emph{CiteULike} (patrocinados por Springer): \url{www.citeulike.org}
	\item Coleção de bibliografia em Ciência da Computação: \url{liinwww.ira.uka.de/bibliography}
	\item Google acadêmico (habilitar bibtex nas preferências): \url{scholar.google.com.br}
\end{itemize}
Lamentavelmente, ainda não existe um mecanismo de verificação ou validação das
informações nessas plataformas. Portanto, é fortemente sugerido validar todas
as informações de tal forma que as entradas bib estejam corretas.  Também, tome
muito cuidado na padronização das referências bibliográficas: ou considere TODOS
os nomes dos autores por extenso, ou TODOS os nomes dos autores abreviados.
Evite misturas inapropriadas.

Exemplos de referências com a tag:
\begin{itemize}
\item @Book: \citep{JW82}.
{\scriptsize\begin{verbatim}
@Book{JW82,
 author   = {Richard A. Johnson and Dean W. Wichern},
 title    = {Applied Multivariate Statistical Analysis},
 publisher= {Prentice-Hall},
 year     = {1983}
}
\end{verbatim}}

\item @Article: \citep{MenaChalco08}.
{\scriptsize\begin{verbatim}
@Article{MenaChalco08,
 author   = {Jesús P. Mena-Chalco and Helaine Carrer and Yossi Zana and 
            Roberto M. Cesar-Jr.},
 title    = {Identification of protein coding regions using the modified 
            {G}abor-wavelet transform},
 journal  = {IEEE/ACM Transactions on Computational Biology and Bioinformatics},
 volume   = {5},
 pages    = {198-207},
 year     = {2008},
}
\end{verbatim}}

\item @InProceedings: \citep{alves03:simi}.
{\scriptsize\begin{verbatim}
@InProceedings{alves03:simi,
 author   = {Carlos E. R. Alves and Edson N. Cáceres and Frank Dehne and 
            Siang W. Song},
 title    = {A Parallel Wavefront Algorithm for Efficient Biological 
            Sequence Comparison},
 booktitle= {ICCSA '03: The 2003 International Conference on Computational Science
            and its Applications},
 year     = {2003},
 pages    = {249-258},
 month    = May,
 publisher= {Springer-Verlag}
}
\end{verbatim}}

\item @InCollection: \citep{bobaoglu93:concepts}.
{\scriptsize\begin{verbatim}
@InCollection{bobaoglu93:concepts,
 author   = {Ozalp Babaoglu and Keith Marzullo},
 title    = {Consistent Global States of Distributed Systems: Fundamental Concepts
            and Mechanisms},
 editor   = {Sape Mullender},
 booktitle= {Distributed Systems},
 edition  = {segunda},
 year     = {1993},
 pages    = {55-96}
}
\end{verbatim}}

\item @Conference: \citep{bronevetsky02}.
{\scriptsize\begin{verbatim}
@Conference{bronevetsky02,
 author   = {Greg Bronevetsky and Daniel Marques and Keshav Pingali and 
            Paul Stodghill},
 title    = {Automated application-level checkpointing of {MPI} programs},
 booktitle= {PPoPP '03: Proceedings of the 9th ACM SIGPLAN Symposium on Principles
            and Practice of Parallel Programming},
 year     = {2003},
 pages    = {84-89}
}
\end{verbatim}}

\item @PhdThesis: \citep{garcia01:PhD}.
{\scriptsize\begin{verbatim}
@PhdThesis{garcia01:PhD,
 author   = {Islene C. Garcia},
 title    = {Visões Progressivas de Computações Distribuídas},
 school   = {Instituto de Computação, Universidade de Campinas, Brasil},
 year     = {2001},
 month    = {Dezembro}
}
\end{verbatim}}

\item @MastersThesis: \citep{schmidt03:MSc}.
{\scriptsize\begin{verbatim}
@MastersThesis{schmidt03:MSc,
 author   = {Rodrigo M. Schmidt},
 title    = {Coleta de Lixo para Protocolos de \emph{Checkpointing}},
 school   = {Instituto de Computação, Universidade de Campinas, Brasil},
 year     = {2003},
 month    = Oct
}
\end{verbatim}}

\item @Techreport: \citep{alvisi99:analysisCIC}.
{\scriptsize\begin{verbatim}
@Techreport{alvisi99:analysisCIC,
 author   = {Lorenzo Alvisi and Elmootazbellah Elnozahy and Sriram S. Rao and
            Syed A. Husain and Asanka Del Mel},
 title    = {An Analysis of Comunication-Induced Checkpointing},
 institution= {Department of Computer Science, University of Texas at Austin},
 year     = {1999},
 number   = {TR-99-01},
 address  = {Austin, {USA}}
}
\end{verbatim}}

\item @Manual: \citep{CORBA:spec}.
{\scriptsize\begin{verbatim}
@Manual{CORBA:spec,
 title    = {{CORBA v3.0 Specification}},
 author   = {{Object Management Group}},
 month    = Jul,
 year     = {2002},
 note     = {{OMG Document 02-06-33}}
}
\end{verbatim}}

\item @Misc: \citep{gridftp}.
{\scriptsize\begin{verbatim}
@Misc{gridftp,
 author   = {William Allcock},
 title    = {{GridFTP} protocol specification. {Global Grid Forum}
            Recommendation ({GFD}.20)},
 year     = {2003}
}
\end{verbatim}}

\item @Misc: para referência a artigo online \citep{fowler04:designDead}.
{\scriptsize\begin{verbatim}
@Misc{fowler04:designDead,
 author   = {Martin Fowler},
 title    = {Is Design Dead?},
 year     = {2004},
 month    = May,
 note     = {Último acesso em 30/1/2010},
 howpublished= {\url{http://martinfowler.com/articles/designDead.html}},
}
\end{verbatim}}

\item @Misc: para referência a página web \citep{FSF:GNU-GPL}.
{\scriptsize\begin{verbatim}
@Misc{FSF:GNU-GPL,
 author   = {Free Software Foundation},
 title    = {GNU general public license},
 year     = {2007},
 note     = {Último acesso em 30/1/2010},
 howpublished= {\url{http://www.gnu.org/copyleft/gpl.html}},
}
\end{verbatim}}

\end{itemize}

         % associado ao arquivo: 'cap-conceitos.tex'
%% ------------------------------------------------------------------------- %%
\chapter{Conclusões}
\label{cap:conclusoes}

Texto texto texto texto texto texto texto texto texto texto texto texto texto
texto texto texto texto texto texto texto texto texto texto texto texto texto
texto texto texto texto texto texto\footnote{Exemplo de referência para página
Web: \url{www.vision.ime.usp.br/~jmena/stuff/tese-exemplo}}.

%------------------------------------------------------
\section{Considerações Finais} 

Texto texto texto texto texto texto texto texto texto texto texto texto texto
texto texto texto texto texto texto texto texto texto texto texto texto texto
texto texto texto texto texto texto. 

%------------------------------------------------------
\section{Sugestões para Pesquisas Futuras} 

Texto texto texto texto texto texto texto texto texto texto texto texto texto
texto texto texto texto texto texto texto texto texto texto texto texto texto
texto texto texto texto texto texto.

Finalmente, leia o trabalho de \citet{alon09:how} no qual apresenta-se
uma reflexão sobre a utilização da Lei de Pareto para tentar definir/escolher
problemas para as diferentes fases da vida acadêmica.  A direção dos novos
passos para a continuidade da vida acadêmica deveriam ser discutidos com seu
orientador.
        % associado ao arquivo: 'cap-conclusoes.tex'

% ---------------------------------------------------------------------------- %

% cabeçalho para os apêndices
\configurapostextual

\chapter{Sequências}
\label{ape:sequencias}

Texto texto texto texto texto texto texto texto texto texto texto texto texto
texto texto texto texto texto texto texto texto texto texto texto texto texto
texto texto texto texto texto texto.


\singlespacing

\renewcommand{\arraystretch}{0.85}
\captionsetup{margin=1.0cm}  % correção nas margens dos captions.
%--------------------------------------------------------------------------------------
\begin{table}
\begin{center}
\begin{small}
\begin{tabular}{|c|c|c|c|c|c|c|c|c|c|c|c|c|} 
\hline
\emph{Limiar} & 
\multicolumn{3}{c|}{MGWT} & 
\multicolumn{3}{c|}{AMI} &  
\multicolumn{3}{c|}{\emph{Spectrum} de Fourier} & 
\multicolumn{3}{c|}{Características espectrais} \\
\cline{2-4} \cline{5-7} \cline{8-10} \cline{11-13} & 
\emph{Sn} & \emph{Sp} & \emph{AC} & 
\emph{Sn} & \emph{Sp} & \emph{AC} & 
\emph{Sn} & \emph{Sp} & \emph{AC} & 
\emph{Sn} & \emph{Sp} & \emph{AC}\\ \hline \hline
 1 & 1.00 & 0.16 & 0.08 & 1.00 & 0.16 & 0.08 & 1.00 & 0.16 & 0.08 & 1.00 & 0.16 & 0.08 \\
 2 & 1.00 & 0.16 & 0.09 & 1.00 & 0.16 & 0.09 & 1.00 & 0.16 & 0.09 & 1.00 & 0.16 & 0.09 \\
 2 & 1.00 & 0.16 & 0.10 & 1.00 & 0.16 & 0.10 & 1.00 & 0.16 & 0.10 & 1.00 & 0.16 & 0.10 \\
 4 & 1.00 & 0.16 & 0.10 & 1.00 & 0.16 & 0.10 & 1.00 & 0.16 & 0.10 & 1.00 & 0.16 & 0.10 \\
 5 & 1.00 & 0.16 & 0.11 & 1.00 & 0.16 & 0.11 & 1.00 & 0.16 & 0.11 & 1.00 & 0.16 & 0.11 \\
 6 & 1.00 & 0.16 & 0.12 & 1.00 & 0.16 & 0.12 & 1.00 & 0.16 & 0.12 & 1.00 & 0.16 & 0.12 \\
 7 & 1.00 & 0.17 & 0.12 & 1.00 & 0.17 & 0.12 & 1.00 & 0.17 & 0.12 & 1.00 & 0.17 & 0.13 \\
 8 & 1.00 & 0.17 & 0.13 & 1.00 & 0.17 & 0.13 & 1.00 & 0.17 & 0.13 & 1.00 & 0.17 & 0.13 \\
 9 & 1.00 & 0.17 & 0.14 & 1.00 & 0.17 & 0.14 & 1.00 & 0.17 & 0.14 & 1.00 & 0.17 & 0.14 \\
10 & 1.00 & 0.17 & 0.15 & 1.00 & 0.17 & 0.15 & 1.00 & 0.17 & 0.15 & 1.00 & 0.17 & 0.15 \\
11 & 1.00 & 0.17 & 0.15 & 1.00 & 0.17 & 0.15 & 1.00 & 0.17 & 0.15 & 1.00 & 0.17 & 0.15 \\
12 & 1.00 & 0.18 & 0.16 & 1.00 & 0.18 & 0.16 & 1.00 & 0.18 & 0.16 & 1.00 & 0.18 & 0.16 \\
13 & 1.00 & 0.18 & 0.17 & 1.00 & 0.18 & 0.17 & 1.00 & 0.18 & 0.17 & 1.00 & 0.18 & 0.17 \\
14 & 1.00 & 0.18 & 0.17 & 1.00 & 0.18 & 0.17 & 1.00 & 0.18 & 0.17 & 1.00 & 0.18 & 0.17 \\
15 & 1.00 & 0.18 & 0.18 & 1.00 & 0.18 & 0.18 & 1.00 & 0.18 & 0.18 & 1.00 & 0.18 & 0.18 \\
16 & 1.00 & 0.18 & 0.19 & 1.00 & 0.18 & 0.19 & 1.00 & 0.18 & 0.19 & 1.00 & 0.18 & 0.19 \\
17 & 1.00 & 0.19 & 0.19 & 1.00 & 0.19 & 0.19 & 1.00 & 0.19 & 0.19 & 1.00 & 0.19 & 0.19 \\
17 & 1.00 & 0.19 & 0.20 & 1.00 & 0.19 & 0.20 & 1.00 & 0.19 & 0.20 & 1.00 & 0.19 & 0.20 \\
19 & 1.00 & 0.19 & 0.21 & 1.00 & 0.19 & 0.21 & 1.00 & 0.19 & 0.21 & 1.00 & 0.19 & 0.21 \\
20 & 1.00 & 0.19 & 0.22 & 1.00 & 0.19 & 0.22 & 1.00 & 0.19 & 0.22 & 1.00 & 0.19 & 0.22 \\ \hline 
\end{tabular}
\caption{Exemplo de tabela.}
\label{tab:tab:F5}
\end{small}
\end{center}
\end{table}

      	% associado ao arquivo: 'ape-conjuntos.tex'

% bibliografia
\backmatter \singlespacing   			% espaçamento simples
\bibliographystyle{bibliografia/plainnat-ime}   % citação bibliográfica textual (pode-se usar o "alpha-ime")
\bibliography{bibliografia/bibliografia}        % associado ao arquivo: 'bibliografia.bib'

% indice remissivo
% Índice remissivo
\index{TBP|see{periodicidade região codificante}}
\index{DSP|see{processamento digital de sinais}}
\index{STFT|see{transformada de Fourier de tempo reduzido}}
\index{DFT|see{transformada discreta de Fourier}}
\index{Fourier!transformada|see{transformada de Fourier}}

\printindex   % imprime o índice remissivo no documento 


\end{document}
