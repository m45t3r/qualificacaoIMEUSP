% Arquivo LaTeX de exemplo de dissertação/tese a ser apresentados à CPG do IME-USP
% 
% Versão 5: Sex Mar  9 18:05:40 BRT 2012
%
% Criação: Jesús P. Mena-Chalco
% Revisão: Fabio Kon e Paulo Feofiloff
% Modificado por Rodrigo Campiolo
%  
% Obs: Leia previamente o texto do arquivo README.md

% ---------------------------------------------------------------------------- %
% Pacotes 
\usepackage[T1]{fontenc}
\usepackage[brazil]{babel}
\usepackage[utf8]{inputenc}
\usepackage[pdftex]{graphicx}           % usamos arquivos pdf/png como figuras
\usepackage{setspace}                   % espaçamento flexível
\usepackage{indentfirst}                % indentação do primeiro parágrafo
\usepackage{makeidx}                    % índice remissivo
\usepackage[nottoc]{tocbibind}  	% acrescentamos a bibliografia/indice/conteudo no Table of Contents
\usepackage{courier}                    % usa o Adobe Courier no lugar de Computer Modern Typewriter
\usepackage{type1cm}            	% fontes realmente escaláveis
\usepackage{listings}                   % para formatar código-fonte (ex. em Java)
\usepackage{titletoc}
%\usepackage[bf,small,compact]{titlesec} % cabeçalhos dos títulos: menores e compactos
\usepackage[fixlanguage]{babelbib}
\usepackage[font=small,format=plain,labelfont=bf,up,textfont=it,up]{caption}
\usepackage[usenames,svgnames,dvipsnames]{xcolor}
\usepackage[a4paper,top=2.54cm,bottom=2.0cm,left=2.0cm,right=2.54cm]{geometry} % margens
%\usepackage[pdftex,plainpages=false,pdfpagelabels,pagebackref,colorlinks=true,citecolor=black,linkcolor=black,urlcolor=black,filecolor=black,bookmarksopen=true]{hyperref} % links em preto
\usepackage[pdftex,plainpages=false,pdfpagelabels,pagebackref,colorlinks=true,citecolor=DarkGreen,linkcolor=NavyBlue,urlcolor=DarkRed,filecolor=green,bookmarksopen=true]{hyperref} % links coloridos
\usepackage[all]{hypcap}                    % soluciona o problema com o hyperref e capitulos
\usepackage[round,sort,nonamebreak]{natbib} % citação bibliográfica textual(plainnat-ime.bst)
\bibpunct{(}{)}{;}{a}{\hspace{-0.7ex},}{,} % estilo de citação. Veja alguns exemplos em http://merkel.zoneo.net/Latex/natbib.php

\fontsize{60}{62}\usefont{OT1}{cmr}{m}{n}{\selectfont}

% ---------------------------------------------------------------------------- %
% Cabeçalhos similares ao TAOCP de Donald E. Knuth
\usepackage{fancyhdr}
\pagestyle{fancy}
\fancyhf{}
\renewcommand{\chaptermark}[1]{\markboth{\MakeUppercase{#1}}{}}
\renewcommand{\sectionmark}[1]{\markright{\MakeUppercase{#1}}{}}
\renewcommand{\headrulewidth}{0pt}

% ---------------------------------------------------------------------------- %
\graphicspath{{./figuras/}}             % caminho das figuras (recomendável)
\frenchspacing                          % arruma o espaço: id est (i.e.) e exempli gratia (e.g.) 
\urlstyle{same}                         % URL com o mesmo estilo do texto e não mono-spaced
\makeindex                              % para o índice remissivo
\raggedbottom                           % para não permitir espaços extra no texto
\fontsize{60}{62}\usefont{OT1}{cmr}{m}{n}{\selectfont}
\cleardoublepage
\normalsize

% ---------------------------------------------------------------------------- %
% Opções de listing usados para o código fonte
% Ref: http://en.wikibooks.org/wiki/LaTeX/Packages/Listings
\lstset{ %
language=Java,                  % choose the language of the code
basicstyle=\footnotesize,       % the size of the fonts that are used for the code
numbers=left,                   % where to put the line-numbers
numberstyle=\footnotesize,      % the size of the fonts that are used for the line-numbers
stepnumber=1,                   % the step between two line-numbers. If it's 1 each line will be numbered
numbersep=5pt,                  % how far the line-numbers are from the code
showspaces=false,               % show spaces adding particular underscores
showstringspaces=false,         % underline spaces within strings
showtabs=false,                 % show tabs within strings adding particular underscores
frame=single,	                % adds a frame around the code
framerule=0.6pt,
tabsize=2,	                % sets default tabsize to 2 spaces
captionpos=b,                   % sets the caption-position to bottom
breaklines=true,                % sets automatic line breaking
breakatwhitespace=false,        % sets if automatic breaks should only happen at whitespace
escapeinside={\%*}{*)},         % if you want to add a comment within your code
backgroundcolor=\color[rgb]{1.0,1.0,1.0}, % choose the background color.
rulecolor=\color[rgb]{0.8,0.8,0.8},
extendedchars=true,
xleftmargin=10pt,
xrightmargin=10pt,
framexleftmargin=10pt,
framexrightmargin=10pt
inputencoding=utf8,             % utf8 for brazillian portuguese
extendedchars=true,
literate=%
{é}{{\'{e}}}1
{è}{{\`{e}}}1
{ê}{{\^{e}}}1
{É}{{\'{E}}}1
{Ê}{{\^{E}}}1
{û}{{\^{u}}}1
{ú}{{\'{u}}}1
{ù}{{\`{u}}}1
{Û}{{\^{U}}}1
{Ú}{{\'{U}}}1
{â}{{\^{a}}}1
{à}{{\`{a}}}1
{á}{{\'{a}}}1
{ã}{{\~{a}}}1
{Á}{{\'{A}}}1
{Â}{{\^{A}}}1
{Ã}{{\~{A}}}1
{ç}{{\c{c}}}1
{Ç}{{\c{C}}}1
{õ}{{\~{o}}}1
{ó}{{\'{o}}}1
{ô}{{\^{o}}}1
{Õ}{{\~{O}}}1
{Ó}{{\'{O}}}1
{Ô}{{\^{O}}}1
{î}{{\^{i}}}1
{í}{{\'{i}}}1
{Î}{{\^{I}}}1
{Í}{{\~{Í}}}1
}

% cabeçalho para as páginas das seções anteriores ao capítulo 1 (frontmatter)
\newcommand{\configurapretextual}{
\frontmatter 
\fancyhead[RO]{{\footnotesize\rightmark}\hspace{2em}\thepage}
\setcounter{tocdepth}{2}
\fancyhead[LE]{\thepage\hspace{2em}\footnotesize{\leftmark}}
\fancyhead[RE,LO]{}
\fancyhead[RO]{{\footnotesize\rightmark}\hspace{2em}\thepage}
\onehalfspacing  % espaçamento
}

% Capítulos do trabalho
\newcommand{\configuracapitulos}{
\mainmatter
% cabeçalho para as páginas de todos os capítulos
\fancyhead[RE,LO]{\thesection}
\singlespacing              % espaçamento simples
%\onehalfspacing            % espaçamento um e meio
}

\newcommand{\configurapostextual}{
\renewcommand{\chaptermark}[1]{\markboth{\MakeUppercase{\appendixname\ \thechapter}} {\MakeUppercase{##1}} }
\fancyhead[RE,LO]{}
\appendix
}


%% Configuração de glossário
\usepackage[portuguese]{nomencl}
\usepackage[acronym,nomain,nonumberlist,nopostdot,nohypertypes={acronym}]{glossaries}
\renewcommand{\glossarymark}[1]{}
\makeglossaries
\glossarystyle{super}		% estilo do glossário
\renewcommand*{\glsgroupskip}{} % espaço entre linhas
\renewcommand*{\acronymname}{Lista de Abreviaturas} 	%Altera título da página
%\renewcommand*{\glssettoctitle}{Lista de Abreviaturas}  %Configura o sumário  para o título
%

%%% FIM das configurações
